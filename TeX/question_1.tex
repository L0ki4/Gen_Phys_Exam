\section{\normalsize Термодинамическая система. Микроскопические и макроскопические параметры. Уравнение состояния (термическое и калорическое). Стационарные, равновесные и неравновесные состояния и процессы. }
\paragraph{Термодинамическая система.}\textbf{Cистема} --- совокупность рассматриваемых тел(\textit{частиц}), которые могут взаимодействовать между собой и с другими телами(\textit{внешняя среда}) посредством обмена веществом и энергией.\\
В термодинамике рассматриваются большие системы, называемые \textbf{термодинамическими системами}.
\paragraph{Микроскопические и макроскопические параметры.}\textbf{Микроскопическое состояние} --- состояние системы, определяемое заданием координат и импульсов (\textit{микропараметры}) всех составляющих систему частиц.\\
\textbf{Макроскопическое состояние} --- состояние системы, характеризующееся небольшим числом макропараметров($P$, $V$, $T$, $\rho$, $\eta$ и т.д.). Макропараметры подразделяются на внутренние и внешние.
\paragraph{Уравнение состояния (термическое и калорическое).}\textbf{Уравнение состояния} --- соотношение, связывающее параметры, описывающие состояние термодинамического равновесия.\\
\textbf{Термодинамическое равновесие} --- состояние, в котором прекращаются все макроскопические процессы: выравниваются давление и температура по объему системы, а скорости прямых и обратных реакций сравниваются.\\
\textbf{Калорическое} уравнение состояния --- зависимость типа $U = U(V,T)$. Пример такого уравнения $PV = (\gamma - 1)U$, где $\gamma$ --- показатель адиабаты, а $U$ - внутренняя энергия всех молекул\\
\textbf{Термическое} уравнение состояния --- зависимость типа $f(P,V,T)=0$\\
\paragraph{Стационарные, равновесные и неравновесные состояния и процессы.}
\textbf{Стационарным состоянием системы} называется состояние, в котором определяющие его параметры не меняются со временем [\textit{в замкнутой системе термодинамическое равновесие это стац. состояние}].\\
В \textbf{равновесном процессе} система непрерывно проходит (бесконечно близкие) равновесные состояния. Все прочие процессы являются неравновесными.\\
\textbf{Равновесным состоянием} является состояние системы, при которым компоненты системы находятся в ТДР, и, как следствие, неизменны их макроскопические параметры. \\
\textbf{Неравновесный процесс} --- процесс, на траектории (\textit{совокупность всех промежуточных состояний}) которого встречаются неравновесные состояния.\\
\textbf{Неравновесное состояние} --- параметры системы меняются от точки к точке с течением времени.\\
\textbf{Обратимым} называют процесс, который может протекать как в прямом, там и в обратном направлении, причем возможно возвращение системы и ее окружения в исходное (макроскопическое) состояние. Если это неосуществимо, то процесс \textbf{необратим}. Неравновесные процессы необратимы. \\
\textbf{Круговой процесс} --- замкнутый равновесный процесс.
