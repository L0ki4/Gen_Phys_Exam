\section{\normalsize Третье начало термодинамики.} Третье начало включает в себя следующие утверждения: 
\begin{enumerate}
	\item при приближении к абсолютному нулю энтропия стремится к конечному предельному значению $S^{(0)}$;
	\item все процессы при абсолютном нуле температур, переводящие систему из одного равновесного состояния в другое, происходят без изменения энтропии.
\end{enumerate}
Последнее утверждение означает, что $S(T)-S^{(0)}=\int_{T_0}^{T}\frac{\delta Q}{dT}<\infty$
при $T\rightarrow0$ и что $\lim\limits_{T\rightarrow0}(S-S^{(0)})$ не зависит от конечного состояния.\\
Объединяя: при приближении к абсолютному нулю приращение энтропии $S-S^{(0)}$ стремится к вполне определенному, конечному пределу, не зависящему от значений, которые принимают все параметры, характеризующие состояние системы($V$, $P$, агрегатное состояние и пр.).\\
Теорема относится только к т.д. равновес. сост. систем.