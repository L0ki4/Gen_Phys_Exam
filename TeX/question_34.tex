\section{\normalsize Флуктуация температуры в заданном объёме. Флуктуация объёма в изотермическом и адиабатическом процессах.}
\paragraph{Флуктуация температуры в заданном объеме.}
Предположим, что рассматриваемая макро- подсистема находится в тепловом контакте с термостатом. 
Считая $T=T\left( V,\varepsilon \right)$ --- независимые переменные\\
$\Delta T=\left( \dfrac {\partial T}{\partial V}\right) _{\varepsilon }\Delta V+\left( \dfrac {\partial T}{\partial \varepsilon }\right)_V \Delta \varepsilon$ . В силу независимости $V$ и $\varepsilon: \overline {\Delta V\Delta \varepsilon }=0 \Rightarrow\\\Rightarrow \overline {\left( \Delta T\right) ^{2}}=\left( \dfrac {\partial T}{\partial v}\right) ^{2}_{\varepsilon }\overline {\left( \Delta v\right) ^{2}}+\left( \dfrac {\partial T}{\partial \varepsilon }\right) ^{2}_{v}\overline {\left( \Delta \varepsilon \right) ^{2}}$\\
Определим $\overline {\left( \Delta \varepsilon \right) ^{2}}$, при условии $W_{i}=\dfrac {1}{z}e^{-\beta \varepsilon _{i}}$\\
$\overline {\left( \Delta \varepsilon \right) ^{2}}=\overline{\varepsilon ^{2}}-\left( \overline {\varepsilon }\right) ^{2}; < \varepsilon  > =\sum _{i}\varepsilon _{i}W_{i}=\dfrac {1}{z}\sum _{i}\varepsilon _{i}e^{-\beta \varepsilon _{i}} =-\dfrac {1}{z}\dfrac {\partial }{\partial \beta }\left( \sum _{i}e^{-\beta \varepsilon _{i}}\right) =-\dfrac {1}{z}\dfrac {\partial z}{\partial \beta }=-\dfrac {\partial }{\partial \beta }\ln z\\
< \varepsilon ^{2} > =\sum _{i}\varepsilon ^{2}_{i}W_{i}=\dfrac {1}{z}\sum _{i}\varepsilon ^{2}_{i}e^{-\beta \varepsilon _{i}}=\dfrac {1}{z}\dfrac {\partial ^{2}}{\partial \beta ^{2}}\sum _{i}e^{-\beta \varepsilon _{i}}=\dfrac {1}{z}\dfrac {\partial ^{2}z}{\partial \beta ^{2}}\Rightarrow\\ \Rightarrow \overline {\left( \Delta \varepsilon \right) ^{2}}=\dfrac {1}{z}\dfrac {\partial ^{2}z}{\partial \beta ^{2}}-\dfrac {1}{z^{2}}\left( \dfrac {\partial z}{\partial \beta }\right) ^{2}=\dfrac {\partial ^{2}\ln z}{\partial \beta ^{2}}=-\dfrac {\partial \overline {\varepsilon }}{\partial \beta }=kT^{2}\dfrac {\partial \overline {\varepsilon }}{\partial T}$\\
Тогда при $V = const : \left( \overline {\varepsilon }\right) ^{2}_{v}=kTc_v\Rightarrow
\overline {\left( \Delta T\right) ^{2}}=\dfrac {1}{c_v^{2}}kT^{2}c_v=\dfrac {kT^{2}}{c_v}
\overline {\left( \Delta T\right) ^{2}}=\dfrac {kT^{2}}{c_v}$

\paragraph{Флуктуаций объема.}
Рассмотрим полную часть газа, окруженную такой же средой, температура которой поддерживается постоянной и равной $Т$. Пусть эта малая часть газа заключена в цилиндре с поршнем. Стенки идеально проводят тепло, трения во время движения поршня нет. 
$<x> = 0 \Rightarrow \Delta V=S_\perp x\Rightarrow \left<\Delta V\right>=S_\perp<x>=0\Rightarrow\\\Rightarrow\sigma^2_V=\left<\left(\Delta V\right)^2\right>=S_\perp^2<x^2>$;\\
$\overline{\Delta \text{П}} = \left<\varkappa(\Delta l_0+x)^2/2+\varkappa\Delta l_0^2/2\right>=<\varkappa x^2/2+\varkappa x\Delta l_0>=\varkappa<x^2>/2+\varkappa \Delta l_0<x>=\\=\varkappa<x^2>/2=<K>=1/2kT$ --- Теория о равном распределении кинетической энергии по степеням свободы\\
$<x^2>=\frac{kT}{\varkappa}\Rightarrow\sigma_V^2=S_\perp^2\frac{kT}{\varkappa};\ \varkappa=\frac{dF_\text{упр.}}{dx}=-\frac{S_\perp dP}{dV/S_\perp}=-S_\perp^2=\Chpr{P}{V}{T}\Rightarrow(\sigma_V^2)_T=S_\perp^2\frac{kT}{\varkappa}=\\=-kT\Chpr{V}{P}{T}\Rightarrow\overline{(\Delta V)^2_T}=-kT\Chpr{V}{P}{T}$\\
При постоянной энтропии : $\overline {\left( \Delta V\right)^{2}_{s}}=-kT\left( \dfrac {\partial V}{\partial P}\right) _{s}$\\
Для идеального газа при $T = const : PV = const \Rightarrow \Chpr{V}{P}{T}=-V/P$, а так как $PV = nkT$, где $n$ - число молекул в $V$, то 
$\overline {\left( \Delta V\right) ^{2}_{T}}=\dfrac {V^{2}}{n}\Rightarrow \sigma _{v_{T}}\sim \dfrac {1}{n}$