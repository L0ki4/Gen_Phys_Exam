\section{\normalsize Влияние флуктуаций на чувствительность измерительных пределов.}
\paragraph{Пружинные весы} Тепловые флуктуации внешней среды и механизма весов приводят к тому, что груз будет совершать хаотические колебания. В результате будет меняться потенциальная энергия пружины:
\[ u=\varkappa(\Delta x)^2/2 \]
где $\Delta x$ --- удлинение, $\varkappa$ --- упругость пружины.
По теореме о равномерном распределении энергии по степеням свободы $\overline{u}=kT/2\Rightarrow\overline{(\Delta x)^2}=kT/\varkappa$\\
Измерение массы возможно, если $x=mg/\varkappa>\sqrt{\overline{(\Delta x)^2}}$\\
Поэтому минимальная масса, которая может быть найдена при однократном измерении:
\[m=\frac{\varkappa}{g}\sqrt{\overline{(\Delta x)^2}}=\frac{\sqrt{kT\varkappa}}{g}\]
\paragraph{Газовый термометр.} Газовый термометр, заполненный идеальным газом, содержит $N$ частиц и имеет объем $V$. Измерение $Т$ производится по изменению объема газа при $P = const$ (давление окружающей среды).\\
Приводя его в контакт с объектом $Т$ возрастает на $\Delta T$, а $V$ на $\Delta V$. В отсутствие флуктуаций: 
\[
\Delta T=\frac{P \Delta V}{kN}=\frac{T}{V}\Delta V \Rightarrow
\]
Измеряя $\Delta V$ находим $\Delta T$, но вследствие теплового движения молекул объем может флуктуировать $\Rightarrow \sigma_V=\frac{V}{\sqrt{N}} \Rightarrow$ для точности термометра при однократном измерении:
\[
\Delta T=\frac{T}{V}\sigma_V=\frac{T}{\sqrt{N}}
\]