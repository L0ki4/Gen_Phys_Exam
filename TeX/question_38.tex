\section{\normalsize Столкновения. Эффективное газокинетическое сечение. Длина свободного пробега. Распределение частиц по длинам свободного пробега.}
Считаем частицы твердыми шариками диаметра $d$. \textbf{Эффективное газокинетическое сечение} --- площадь поперечного сечения цилиндра радиусом, равным максимальному расстоянию между центрами сталкивающихся молекул.
\[
\sigma =\pi d^2
\]
\[
\lambda\sigma n=1\rightarrow\lambda=\frac{1}{\sigma n}
\]
$\overrightarrow{v_\text{отн.}}=\overrightarrow{v}-\overrightarrow{v}',\ \overline{v_\text{отн.}^2}=\overline{v^2}+\overline{v'^2}-2(\overline{\overrightarrow{v}\overrightarrow{v}'})=\overline{v^2},\ \overline{v_\text{отн.}}\simeq\sqrt{2}v$
\[
\lambda=\frac{1}{\sqrt{2}\sigma n}
\]
\paragraph{Частота столкновения молекул газа}. $\lambda=\frac{1}{\sqrt{2}n\sigma}\Rightarrow\tau=\frac{1}{\sqrt{2}n\overline{v}\sigma}$--- время свободного пробега => за единицу времени молекула испытает $\frac{1}{\tau}$ столкновений, а так как в единице объема $n$ молекул, то всего за единицу времени она испытает число столкновений:
\[
f=\frac{n}{2\tau}=\frac{n^2\overline{v}\sigma}{\sqrt{2}}
\]
$P=nkT,\ \sigma=\pi d^2,\ \overline{v}=\sqrt{\frac{8kT}{\pi m}}$ получаем
\[
f=2\sqrt{\frac{\pi}{m}}\frac{P^2d^2}{(kT)^{3/2}}
\]
\paragraph{Распределение молекул по длинам свободного пробега.} Будем искать распределение вероятностей различных значений пути, проходимого молекулами до столкновения. В слое толщиной $dx$ одна молекула испытывает $n\sigma dx$ столкновений. Проведем $N_0$ испытаний, то есть запустим в среду $N_0$ молекул.\\
Пусть N молекул пройдет трассу длиной $x$ без столкновений, тогда убыль их числа на последующем участке $dx$ составит $Nn\sigma dx$, т.е. 
\[
dN=-Nn\sigma dx=-N\frac{dx}{\lambda}\Rightarrow N(x)=N_0\exp{-x/\lambda}
\]
$dN_\text{расс.}=-dN=N_0\exp(-x/\lambda)dx/\lambda$ --- число молекул, рассеявшихся на участке $x\div x+dx$. Поэтому вероятность прохождения молекулой такого пути (без столкновений), то есть того, что длина свободного пробега равна $x$.
\begin{equation}
\label{eq:38}
dW(x)=\frac{dN_\text{расс.}}{N_0}=\exp(-x/\lambda)dx/\lambda
\end{equation}
Причем, $\int_{0}^{\infty}dW(x)=1$
\begin{align*}
\overline{x}=\int_{0}^{\infty}x\exp(-x/l)dx/l=l\\
\overline{x^2}=\int_{0}^{\infty}x^2\exp(-x/l)dx/l=2l^2\\
\mathbb{D}x=\overline{(x-\overline{x})^2}=\overline{x^2}-\overline{x}^2=l^2
\end{align*}
 Таким образом $\lambda$ имеет смысл длины свободного пробега, а распределение вероятностей длин свободного пробега дается формулой \eqref{eq:38}.