\section{\normalsize Термодинамическая система. Микроскопические и макроскопические параметры. Уравнение состояния (термическое и калорическое). Стационарные, равновесные и неравновесные состояния и процессы. }
\paragraph{Термодинамическая система.}\textbf{Cистема} --- совокупность рассматриваемых тел(\textit{частиц}), которые могут взаимодействовать между собой и с другими телами(\textit{внешняя среда}) посредством обмена веществом и энергией.\\
В термодинамике рассматриваются большие системы, называемые \textbf{термодинамическими системами}.
\paragraph{Микроскопическое состояние и макроскопическое состояние.}\textbf{Микроскопическое состояние} --- состояние системы, определяемое заданием координат и импульсов (\textit{микропараметры}) всех составляющих систему частиц.\\
\textbf{Макроскопическое состояние} --- состояние системы, характеризующееся небольшим числам макропараметров($P$, $V$, $T$, $\rho$, $\eta$ и т.д.). Макропараметры подразделяются на внутренние и внешние.
\paragraph{Уравнение состояния (термическое и калорическое).}\textbf{Уравнение состояния} --- соотношение, связывающее параметры, описывающие состояние термодинамического равновесия. \textbf{Термодинамическое равновесие} --- состояние, в котором прекращаются все макроскопические процессы: выравниваются давление и температура по объему системы, а скорости прямых и обратных реакций сравниваются.\\
\textbf{Калорическое} уравнение состояния --- зависимость типа $U = U(V,T)$. Пример такого уравнения $PV = (\gamma - 1)U$, где $\gamma$ --- показатель адиабаты, а $U$ - внутренняя энергия всех молекул\\
\textbf{Термическое} уравнение состояния --- зависимость типа $f(P,V,T)=0$\\
\paragraph{Стационарные, равновесные и неравновесные состояния и процессы.}
\textbf{Стационарным состоянием системы} называется состояние, в котором определяющие его параметры не меняются со временем [\textit{в замкнутой системе термодинамическое равновесие это стац. состояние}].\\
\textbf{Равновесный процесс} --- процесс, в каждый момент которого система находится вблизи термодинамического равновесия, отвечающего суммарному воздействию на систему.\\
\textbf{Неравновесный процесс} --- процесс, на траектории (\textit{совокупность всех промежуточных состояний}) которого встречаются неравновесные состояния.\\
\textbf{Неравновесное состояние} --- параметры системы меняются от точки к точке с течением времени.\\
Процесс называется \textbf{обратимым}, если он может быть проведен в обратном направлении через те же промежуточные состояния, что и прямой, причем в остальных телах таких изменений не происходит. Если это неосуществимо, то процесс \textbf{необратим}. \\
\textbf{Круговой процесс} --- процесс, начинающийся и заканчивающийся в одной и той же точке, т.е. его траектория замкнута.