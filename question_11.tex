\section{\normalsize  Термодинамические потенциалы. Метод получения соотношений Максвелла (соотношений взаимности). }
\paragraph{Термодинамические потенциалы.} \textbf{Термодинамические потенциалы} --- функции определённых наборов термодинамических параметров, позволяющие находить все термодинамические характеристики системы, как функции этих параметров.
\paragraph{Метод получения соотношений Максвелла (соотношений взаимности).} У Лёши Шевцова он представлен на примере вывод одного из потенциалов. В Сивухине похожая ситуация, так что пока что опустим этот пункт, а в следующем билете вы увидите, что это за метод и в чем его суть.
