\section{\normalsize Метастабильные состояния: переохлажденный пар, перегретая жидкость. Устойчивость состояний. Правило Максвелла.}
\paragraph{Метастабильные состояния: переохлажденный пар, перегретая жидкость. Устойчивость состояний.} При специальных условия могут быть реализованы участки AG --- перенасыщенный пар и LB --- перегретая жидкость (см. рис. 3). Эти состояния называют \textbf{метастабильными}. Каждое существует, пока его менее устойчивая фаза не граничит с другой --- более устойчивой. Например, перенасыщенный пар переходит в насыщенный, если в него попадет капля воды.
\paragraph{Правило Максвелла.}Положение горизонтального участка определяется с помощью равенства Клазиуса $\oint\dfrac{\delta Q}{T}=0$. Из G в L можно перейти двумя путями: по GCL двухфазного состояния и по теоретической изотерме физически однородного вещества GACBL, содержащей неустойчивый участок ACB. Применим равенство Клаузиуса к квазистатическому круговому процессу GCLBCAG: $T=const\Rightarrow\oint\delta Q=0$, кроме того $\delta Q~=~dU~+~PV,\quad \\\oint dU=0\Rightarrow\oint PdV=0$ или $\int_{\text{GCL}}PdV+\int_{\text{LBCAG}}PdV=0$ или $\int_\text{LCG}PdV=\int_\text{LBCAG}PdV$\\
Значит площади QLGR и QLBCAGR равны, значит LG надо проводить так, чтобы $$S_\text{LBC}=S_\text{CAG}$$ Это и есть \textbf{правило Максвелла.}
