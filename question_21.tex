\section{\normalsize Внутренняя энергия и энтропия газа Ван-дер-Ваальса. Равновесное и неравновесное расширение газа Ван-дер-Ваальса в теплоизолированном сосуде.
}
\paragraph{Внутренняя энергия и энтропия газа Ван-дер-Ваальса.} Рассмотрим $U=U(T,V)$, тогда $dU=\chpr{U}{T}{V}dT+\chpr{U}{V}{T}dV=C_VdT+\left(T\chpr{P}{T}{V}-P\right)dV$. Для $\nu$ = 1 моль, предполагая $C_V=const,\ dU=C_VdT+\dfrac{a}{V^2}dV$
$$U=C_VT-\dfrac{a}{V}$$
С ростом объема и, следовательно, расстояния между молекулами (при $T=const$) внутренняя энергия газа растет.\\
Рассмотрим $S=S(T,V)$, тогда $dS=\chpr{S}{T}{V}dT+\chpr{S}{V}{T}dV=\dfrac{C_V}{T}dT+\chpr{P}{T}{V}dV$. Для $\nu=1$ моль, $C_V=const,\ dS=\dfrac{C_V}{T}dT+\dfrac{R}{V-b}dV$ 
$$S=S_0+C_V\ln\left(\dfrac{T}{T_0}\right)+R\ln\left(\dfrac{V-b}{V_0-b}\right)$$
\paragraph{Равновесное и неравновесное расширение газа Ван-дер-Ваальса в теплоизолированном сосуде.} Рассмотри \textbf{свободное расширение газа в вакуум.} Пусть в начальный момент газ Ван-дер-Ваальса находился в сосуде, занимая в нем объем $V$. После удаления перегородки газ получил возможность свободно расшириться до объема $V_2(V_2>V_1)$. Считая, что сосуд окружен теплоизолированной оболочкой найдем $\Delta T$ после установления равновесия. Поскольку $\delta Q=0$ и $\delta A=0$, то $dU=0$, а для газа Ван-дер-Ваальса\\ $U_{1,2}=C_VT_{1,2}-\dfrac{a}{V_{1,2}}\Rightarrow\Delta T=T_2-T_1=-\dfrac{a}{C_V}\left(\dfrac{1}{V_1}-\dfrac{1}{V_2}\right)<0$ значит газ в данном процессе \textbf{охлаждается}.\\
При расширении газа работа совершается против сил притяжения молекул. Эта работа производится за счет кинетической энергии и, значит, сопровождается охлаждением газа
