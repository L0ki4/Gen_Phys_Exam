\section{\normalsize Микро- и макросостояния. Статистический вес. Распределение Гиббса.}
\paragraph{Микро- и макросостояния.} \textbf{Микроскопическое состояние} --- состояние системы, определяемое заданием координат и импульсов.\\
\textbf{Макроскопическое состояние} --- состояние системы, характеризующееся небольшим числом макропараметров($P$, $V$, $T$, $\rho$, $\eta$ и т.д.).\\
Одно макросостояние может быть реализовано большим числом микросостояний за счете перестановки частиц, не меняющей наблюдаемого состояния.
\paragraph{Статистический вес.} Если система состоит из $N$ частиц, тогда $6N$ чисел полностью характеризует её состояние ($3N$ координат и $3N$ импульсов). Разобьем 6-ти мерный фазовый объем фазовый объем $V$ на $m$ ячеек $V_1,\ldots,V_m$. Каждая молекула находится в какой-то ячейке $N_1,\ldots,N_m$\\
$P_i=\frac{V_i}{V}$ --- вероятность попасть в $i$-ую ячейку для каждой частицы, если в $i$-ой ячейке $N_i$ молекул, то $P_i=\left(\frac{V_i}{V}\right)^{N_i}.$ Все микросостояния $P_1^{N_1}\times\ldots\times P_m^{N_m}$, а если $V_i=v,~\forall~i=~\overline{1,m}$, то $\left(\frac{v}{V}\right)^N$. Все частицы тождественны, значит число способов реализовать данное макросостояние: $G=\frac{N!}{N_1!\ldots N_m!}$ --- \textbf{Статистический вес}.\\
\textbf{Статистический вес} --- число равновероятностных микросостояний, каждое из которых реализует данное макросостояние, то есть термодинамическая вероятность данного распределения $P=GP_1^{N_1}\times\ldots\times P_m^{N_m}$, а в случае одинаковых ячеек $\left(\frac{v}{V}\right)^N=const\Rightarrow P=G$ (с точностью до константы).
\paragraph{Распределение Гиббса.} Он пока не зашел, если что 80-82 страницы Кириченко. Завтра перенесу.