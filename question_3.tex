\section{\normalsize Идеальный газ. Связь давления и температуры идеального газа с кинетической энергией его молекул. Уравнение состояния идеального газа.}
\paragraph{Идеальный газ.} \textbf{Идеальный газ} --- газ, расстояние между молекулами которого настолько велико, что их взаимодействием можно пренебречь, а его внутренняя энергия --- кинетическая энергия частиц.
\paragraph{Связь давления и температуры идеального газа с кинетической энергией его молекул.} Число молекул со скоростью $v$ в единице объема --- $n(v)$ и импульс одной молекулы $p_x=mv_x$. Тогда импульс переданный стенке молекулой --- $2p_x$. Число молекул, которые долетают до стенки за $dt: \frac{1}{2}n(v)Sv_xdt \then$ суммарное изменение импульса $\Delta p = p_x nSv_xdt \then$ полный импульс по всему группам молекул: $$\sum_{v} p_xnSv_xdt = F_x dt \then P = \dfrac{F_x}{S}= \sum_{v}p_xnv_x=n\overline{v_x p_x}=2n\dfrac{\overline{mv_x^2}}{2},\; n = \sum_v n(v)$$
Вследствие изотропии газа $\overline{v_xp_x} =\overline{v_yp_y} =\overline{v_zp_z} =\frac{1}{3} \overline{vp} \then P = \dfrac{2}{3}n\dfrac{\overline{mv^2}}{2}$,  так как $E_\text{кин.} =\\= 3/2kT \then P=nkT$
\paragraph{Уравнение состояния идеального газа.} \textbf{Уравнение состояния вещества} --- соотношение, связывающее параметры, описывающие состояния термодинамического равновесия вещества.\\
\textbf{Для идеального газа} $PV = \nu RT$, где $\nu = \dfrac{m}{\mu}$, $R$ --- универсальная газовая постоянная.
