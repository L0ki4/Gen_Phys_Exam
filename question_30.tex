\section{\normalsize Статистическое определение энтропии. Аддитивность энтропии. Закон возрастания энтропии.} 
\paragraph{Статистическое определение энтропии.} $S=k_\text{Б.}\ln G$.
\paragraph{Аддитивность энтропии.} Разобьем систему на 2 подсистемы со статистическими весами $G_1$ и $G_2$. Если взаимодействие между ними слабое, то микросостояния у них меняются независимо, тогда $G=G_1\cdot G_2$	--- статистический вес системы. Тогда $S=k\ln G_1+ k\ln G_2=S_1+S_2$ --- свойство аддитивности. В общем случае: $S=k\ln G=k \ln N!-\sum_{i=1}^{m}k\ln N_i!=\\=\left|\blacktriangleright\ln N!\simeq N\ln N-N\blacktriangleleft\right|=kN\ln N-\cancel{kN}-k\sum_{i=1}^{m}N_i\ln N_i+\cancel{k\sum_{i=1}^{m}N_i}=kN\left(\ln N-\sum_{i=1}^{m}\frac{N_i}{N}\ln N_i\right)=\\=kN\left(\cancel{\ln N}-\sum_{i=1}^{m}W_i\ln W_i-\cancel{\ln N\sum_{i=1}^{m}W_i}\right)=-kN \sum_{i=1}^{m}W_i\ln W_i$
\[
S=-kN\sum_{i=1}^{m}W_i\ln W_i
\]
\paragraph{Закон возрастания энтропии.} Среди всех направлений эволюции системы предположительным является то, при котором вероятность состояния оказывается наибольшей, тогда
\begin{enumerate}
	\item с наибольшей вероятностью энтропия замкнутой системы не убывает: $\frac{dS}{dt}\geqslant0$;
	\item в состоянии термодинамического равновесия (наиболее вероятное) энтропия максимальна $S=S_{max}; \frac{dS}{dT}=0$
\end{enumerate}
Хотя, если энтропия в какой-то момент достигла своего максимума, то почти наверняка в следующий момент времени она уменьшится. Таким образом энтропия системы, находящейся с макро- точки зрения в состоянии термодинамического равновесия, совершает небольшие колебания.