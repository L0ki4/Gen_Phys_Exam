\section{\normalsize Флуктуации. Зависимость флуктуаций от числа частиц, составляющих систему. Флуктуация числа частиц в заданном объёме.}
\textbf{Флуктуациями} называют случайные отклонения физических величин от среднего значения.
$f$ - некоторая случайная величина, 
$\overline{f}$ - ее среднее значение, 
$\Delta{f} = f - \overline{f}$ - флуктуация, 
$\sigma = \sqrt{\overline{(\Delta{f})^{2}}}$ - среднеквадратичная флуктуация
$\delta_{f} = \frac{\sqrt{\overline{(\Delta{f})^{2}}}}{\overline{f}}$ - относительная среднеквадратичная флуктуация.\newline
\paragraph{Флуктуация числа частиц в заданном объеме.}
Рассмотрим газ из $N$ частиц, занимающий обьем $V$. Выделим в нем маленький обьемчик $v$. Вероятность каждой частицы оказаться в этом обьемчике $p = \frac{v}{V}$. Среднее число частиц в выделенном обьемчике $\overline{n} = \frac{Nv}{V}$.
\newline
Вероятность попадания $n$ частиц в обьемчик $v$ находится по схеме Бернулли
\[
W(n) = {C_{N}^{n}}p^{n}(1-p)^{N-n}
\]
Поскольку выделенный обьем мал по сравнению со всем обьемом газа, а число частиц очень велико, считаем, что $p<<1$, по теореме Муавра-Лапласа, распределение этой величины - нормальное с дисперсией $\sigma^{2}$ где $\sigma = \sqrt{Np(1-p)}$
\[Np = \overline{n},\space  1-p\approx1\space \Rightarrow \sigma = \sqrt{\overline{n}}\]
тогда \[\delta_{n} = \frac{\sigma}{\overline{n}} = \frac{1}{\sqrt{\overline{n}}}\]
\paragraph{Зависимость флуктуация от числа частиц.} Аддитивная величина $F = \sum_{i}^{} f_{i}$,
Считаем все частицы одинаковыми, поэтому $\overline{F} = N\overline{f}$
\[\overline{F^{2}} = \overline{{(\sum{f_i})}^{2}} = \sum{f_{i}^{2}} + \sum_{i\ne k}^{} \overline{f_{i}f_{k}}\]
из независимости величин $f_{i}$ следует $\overline{f_{i}f_{k}} = (\overline{f})^{2}$
\[\overline{F^{2}} = N\overline{f^{2}} + N(N-1)(\overline{f})^{2}\]

\[\overline{(\Delta F)^{2}} = \overline{F^{2}} - (\overline{F})^{2} = N\overline{f^{2}} + N(N-1)(\overline{f})^{2} - N^{2}(\overline{f})^{2} = N(\overline{f^{2}} - (\overline{f})^{2}) = N\sigma_{f}^{2}\]

то есть 
\[\sqrt{\overline{(\Delta F)^{2}}} = \sqrt{N}\sigma_{f}\]

Теперь рассмотрим интенсивную(не зависящую от обьема) величину
\[G = \frac{1}{N}\sum g_{i}\]
аналогичным образом для нее получаем

\[\overline{(\Delta G)^{2}} = \overline{G^{2}} - (\overline{G})^{2} = \frac{1}{N^{2}}(N\overline{g^{2}} + N(N-1)(\overline{g})^{2}) - (\overline{g})^{2} = \frac{1}{N}(\overline{g^{2}} - (\overline{g})^{2}) = \frac{\sigma_{g}^{2}}{N}\]
то есть 
\[\sqrt{\overline{(\Delta G)^{2}}} = \frac{\sigma_{g}}{\sqrt{N}}\] 