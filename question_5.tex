\section{ \normalsize Теплоемкость. Теплоемкости $C_V$ и $C_P$. Теплоемкости идеального газа при постоянном объеме и давлении, соотношение Майера.}
\paragraph{Теплоемкость.}  \textbf{Теплоемкостью тела} называется отношение бесконечно малого количества теплоты $\delta Q$, полученного телом, к соответственному приращению его температуры $dT$ $$C = \dfrac{\delta Q}{dT}$$
\paragraph{Теплоемкости $C_V$ и $C_P$.} \textbf{Удельная теплоемкость $c$} --- теплоемкость в расчете на единицу массы.\\
\textbf{Молярная теплоемкость $C_\mu$} --- теплоемкость в расчете на 1 моль.
\paragraph{Теплоемкости идеального газа при постоянном объеме и давлении.}
$$dU = \left(\dfrac{\partial U}{\partial T}\right)_V dT + \left(\dfrac{\partial U}{\partial V}\right)_T dV  \then C = \dfrac{dU + PdV}{dT} = \left(\dfrac{\partial U}{\partial T}\right)_V + \left[\left(\dfrac{\partial U}{\partial V}\right)_T +P\right]\dfrac{dV}{dT}$$
При постоянном \textbf{объеме}: $C_v = \left(\dfrac{\partial U}{\partial T}\right)_V$\\
При постоянном \textbf{давлении}: $C_P= \left(\dfrac{\partial U}{\partial T}\right)_V + \left[\left(\dfrac{\partial U}{\partial V}\right)_T +P\right]\left(\dfrac{dV}{dT}\right)_P$\\
\paragraph{Cоотношение Майера.} Для \textbf{идеального газа} $\left(\dfrac{\partial U}{\partial V}\right)_T = 0$
$$C_v = \left(\dfrac{\partial U}{\partial T}\right)_V, C_P = \left(\dfrac{\partial U}{\partial T}\right)_V + P\left(\dfrac{dV}{dT}\right)_P \then C_P-C_V = P\left(\dfrac{dV}{dT}\right)_P\left.\right|_{PV=RT}=P\cdot\dfrac{R}{P}=R$$
\textbf{Cоотношение Майера} --- $C_P-C_v = R$
