\section{\normalsize Адиабатический и политропический процессы. Уравнение адиабаты и политропы идеального газа.}
\paragraph{Адиабатический и политропический процессы.} \textbf{Адиабатическим} называется процесс, происходящий в теплоизолированной системе ($\partial Q = 0$)\\
\textbf{Политропическим} называется процесс, происходящий при постоянной теплоемкости ($C=const$)\\
Примеры политропических процессов:
\begin{enumerate}
	\item Адиабатический: $C=0,\,n=\gamma,\,PV^\gamma=const$
	\item Изобарический: $C=C_p,\,n=0,\,P=const$
	\item Изохорический: $C=C_v,\,n=\infty,\,V=const$
	\item Изотермический: $C=\infty,\,n=1,\,T=const$
\end{enumerate}
\paragraph{Уравнение адиабаты и политропы идеального газа.}
$$ \delta Q = C_VdT + PdV =0;\; T =\dfrac{PV}{R} \then dT = \dfrac{d(PV)}{R}=\dfrac{PdV+VdP}{R}=\dfrac{PdV+VdP}{C_P-C_V}\then $$
$$\then C_V \dfrac{PdV+VdP}{C_P-C_V}+PdV = 0 \Leftrightarrow C_VPdV+C_VVdP+C_PPdV-C_VPdV=0$$
Введем $\gamma=\dfrac{C_P}{C_V}$, тогда $\gamma PdV+VdP = 0 \Leftrightarrow \gamma d(\ln V)+d(\ln P) = 0$\\
У идеального газа $C_V = const,\,C_p=const\then\gamma=const\then d(\ln PV^\gamma)=0$\\
В итоге получаем $PV^\gamma=const$ --- уравнение \textbf{Пуассона.}\\

Теперь выведем \textbf{уравнение политропы}.
$$ \delta Q = CdT = C_VdT+PdV \Leftrightarrow (C-C_V)\dfrac{dT}{T}=R\dfrac{dV}{V} \Leftrightarrow$$
$$\Leftrightarrow (C-C_V)\ln T = R\ln V + const \Leftrightarrow \ln T = \ln \left(V^{\tfrac{R}{C-C_V}}\right) + const \Leftrightarrow TV^{-\tfrac{R}{C-C_V}}=const $$
Введем показатель политропы $n = \dfrac{C-C_P}{C-C_V}$, тогда $TV^{n-1}=const$ --- \textbf{уравнение политропы}.
