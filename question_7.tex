\section{\normalsize Скорость звука в газах. Скорость истечения газа из отверстия.}
\paragraph{Скорость звука в газах.} Газы и жидкости обладают упругостью объема, но не формы. Поэтому в них могут распространяться только продольные волны. Фазовая скорость продольных волн в бесконечной упругой среде равна $c_\text{зв.}=\sqrt{E_1/\rho}$, где $E_1$ --- модуль одностороннего растяжение. Из закона Гука $E_1=\sigma/\varepsilon$, $\varepsilon=\Delta l/l=\Delta V/V$. Заменяя $\sigma \rightarrow -\Delta P$, рассматривая бесконечно малые изменения объема $\Delta V \rightarrow dV$ и давления $\Delta P \rightarrow dP$ и учитывая, что $V\sim\rho^{0-1}$ получаем $E_1=-V\partial P/\partial V=\rho \partial P/\partial\rho$. Отсюда $c_\text{зв.}=\sqrt{\partial P/\partial\rho}$\\
Если за время прохождения звука на расстояние порядка длины волн $\lambda$ тепло не успевает выйти за пределы объема $V\sim\lambda^3$, то такой процесс адиабатический и $c_\text{зв.}=\sqrt{(\partial P/\partial\rho)_{\text{адиаб.}}}$\\
Для идеального газа из уравнения адиабаты $P\sim\rho^\gamma$, откуда $(\partial P/\partial\rho)_{\text{адиаб.}}=\gamma P/\rho$. Из уравнения состояния $P=\rho RT/\mu$. Отсюда следует выражение для скорости звука.
\[c_\text{зв.}=\sqrt{\gamma {/\rho}}=\sqrt{\gamma RT/\mu}  \]
\paragraph{Истечение газа из отверстия.} Исследуем адиабатическое ламинарное течение. Пусть изначально газ находился в сосуде при давлении $P_1$ и температуре $T_1$, после он истекает в среду с температурой $T_2$ и давлением $P_2$, известны все эти величины кроме $T_2$.\\
Уравнение Бернулли:
\begin{equation}
\label{bernuli} 
\dfrac{v_1^2}{2}+\dfrac{P_1}{\rho_1}+gh_1+u_1=\dfrac{v^2_2}{2}+\dfrac{P_2}{\rho_2}+gh_2+u_2,\,
\end{equation}  $gh=const$, т.к. не меняется существенно вдоль трубки тока.  Введем $H=I=U+PV$ --- энтальпия. $i = u+Pv_\text{уд.}$ --- удельная энтальпия. Подставим i в (\ref{bernuli}) $$\dfrac{v_1^2}{2}+i_1=\dfrac{v_2^2}{2}+i_2$$
Если сосуд большой, а отверстие мало, то можно принять, что скорость газа в сосуде $v_1=0 \then v_2=\sqrt{2(i_1-i_2)}$. \\
В случае идеального газа ($C_V=const$): $i=u+\dfrac{P}{\rho}=\dfrac{C_VT}{\mu}+\dfrac{RT}{\mu}=\dfrac{C_PT}{\mu}\then$\\
$\then v =\sqrt{\dfrac{2}{\mu}C_P(T_1-T_2)}$\\
\textbf{Вычислительная формула}: $\dfrac{P_1^{\gamma-1}}{T_1^\gamma}=\dfrac{P_2^{\gamma -1}}{T_2^\gamma}\Leftrightarrow T_2=T_1\left(\dfrac{P_2}{P_1}\right)^{\tfrac{\gamma-1}{\gamma}}\\$
$$v =\sqrt{\dfrac{2}{\mu}C_PT_1\left[1-\left(\dfrac{P_2}{P_1}\right)^{\tfrac{\gamma -1}{\gamma}}\right]}$$\\
Максимальная скорость достигается при истечении в вакуум: $v_\text{вак.}=\sqrt{\dfrac{2}{\mu}C_PT}$ или более точно
$v_\text{вак.}=\sqrt{\dfrac{2}{\mu}\dfrac{\gamma}{\gamma-1}RT}=\sqrt{\dfrac{2}{\gamma -1}}c_\text{зв.}$
