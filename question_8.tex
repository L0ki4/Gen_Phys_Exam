\section{\normalsize Цикл Карно. КПД машины Карно. Теоремы Карно.}
\paragraph{Цикл Карно.} \textbf{Тепловая машина} --- устройство, преобразующее теплоту в работу или обратно и действующее строго периодически.\\
\textbf{Машина Карно} --- тепловая машина, работающая между двумя резервуарами с $T_1$ и $T_2$, причем $T_2<T_1$, по обратимому циклу, состоящему из двух изотерм и двух адиабат (циклу Карно).
\paragraph{КПД машины Карно.} \textbf{КПД тепловой машины} --- отношение работы, произведенной машиной за один цикл, к теплоте, поглощенной в ходе рассматриваемого цикла.
$$\eta = \dfrac{A}{Q_\text{н}}=\dfrac{Q_1-Q_2}{Q_1}=1-\dfrac{Q_2}{Q_1}<1$$
Рассчитаем \textbf{КПД машины Карно.} Рабочее тело - идеальный газ.\\
$Q_{12}=\delta U_{12}+A_{12}=RT_1 \ln\left(\dfrac{V_2}{V_1}\right) ,\;\; Q_{34}'=-Q_{34}=-A_{34}=-RT_2\ln\left(\dfrac{V_4}{V_3}\right)$\\
$T_1V_2^{\gamma-1}=T_2V_3^{\gamma -1}\then\dfrac{T_1}{T_2}=\left(\dfrac{V_3}{V_2}\right)^{\gamma-1},\;\;T_2V_4^{\gamma-1}=T_1V_1^{\gamma -1}\then\dfrac{T_1}{T_2}=\left(\dfrac{V_4}{V_1}\right)^{\gamma-1} $\\
$\dfrac{V_3}{V_2} = \dfrac{V_4}{V_1} \then Q_{34}=RT_2 \ln \left(\dfrac{V_2}{V_1}\right),$ тогда $\eta = 1 -\dfrac{Q_{34}'}{Q_{12}}=1-\dfrac{T_2}{T_1}$
\paragraph{Теоремы Карно.} На эту тему существует отличная 
\href{https://mipt.ru/education/chair/physics/S_II/method/Carnot.pdf}{методичка В.С. Булыгина} (надпись кликабельна)