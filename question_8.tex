\section{\normalsize Цикл Карно. КПД машины Карно. Теоремы Карно.}
\paragraph{Цикл Карно.} \textbf{Тепловая машина} --- устройство, преобразующее теплоту в работу или обратно и действует строго периодически.\\
\textbf{Машина Карно} --- тепловая машина, работающая между двумя резервуарами с $T_1$ и $T_2$, причем $T_2<T_1$, по обратимому циклу, состоящему из двух изотерм и двух адиабат (циклу Карно).
\paragraph{КПД машины Карно.} \textbf{КПД тепловой машины} --- отношение работы, произведенной машиной за один цикл, к теплоте, поглощенной в ходе рассматриваемого цикла.
$$\eta = \dfrac{A}{Q_\text{н}}=\dfrac{Q_1-Q_2}{Q_1}=1-\dfrac{Q_2}{Q_1}<1$$
Рассчитаем \textbf{КПД машины Карно.} Рабочее тело - идеальный газ.\\
$Q_{12}=\delta U_{12}+A_{12}=RT_1 \ln\left(\dfrac{V_2}{V_1}\right) ,\;\; Q_{34}'=-Q_{34}=-A_{34}=-RT_2\ln\left(\dfrac{V_4}{V_3}\right)$\\
$T_1V_2^{\gamma-1}=T_2V_3^{\gamma -1}\then\dfrac{T_1}{T_2}=\left(\dfrac{V_3}{V_2}\right)^{\gamma-1},\;\;T_2V_4^{\gamma-1}=T_1V_1^{\gamma -1}\then\dfrac{T_1}{T_2}=\left(\dfrac{V_4}{V_1}\right)^{\gamma-1} $\\
$\dfrac{V_3}{V_2} = \dfrac{V_4}{V_1} \then Q_{34}=RT_2 \ln \left(\dfrac{V_2}{V_1}\right),$ тогда $\eta = 1 -\dfrac{Q_{34}'}{Q_{12}}=1-\dfrac{T_2}{T_1}$
\paragraph{Теоремы Карно.} \textbf{Первая теорема Карно:} КПД тепловой машины, работающей по циклу Карно, зависит только от температур $T_1$ и $T_2$ нагревателя и холодильника, но не зависит от устройства машины, а также от вида использованного рабочего вещества.\\
\textbf{Доказательство.} Рассмотрим две машины Карно с общим нагревателем $T_1$ и холодильником $T_2$. Пусть КПД первой $\eta$, второй --- $\eta'$ и $\eta > \eta'$. \\
Пусть в результате $m$ циклов первая машина совершила работу $A=Q_1-Q_2$, например подняв груз. Используем потенциальную энергию груза для запуска второй машины в обратном направлении, тогда в результате $m'$ циклов над этой машиной будет совершена работа $A'=Q_1'-Q_2'\then$ за $m+m'$ циклов нагреватель отдал $Q_1-Q_1'$, холодильник отдал $Q_2'-Q_2$, а совершенная работа $A-A'=(Q_1-Q_2)-(Q_1'-Q_2')=\eta Q_1-\eta'Q_1'$\\
Используя постулат Томпсона--Планка о невозможности кругового процесса единственным результатом которого было бы совершение работы за счет охлаждения теплового резервуара выберем $m$ и $m'$ так, что $Q_1-Q_1'=0$.\\
Т.к. $Q_1=mq_1,\,Q_1'=m'q_1'$, где $q_1$ и $q_1'$ - теплота за 1 цикл; если $q_1$ и $q_1'$ соизмеримы, то всегда существуют $m$ и $m':\;Q_1-Q_1'=0$. Если нет, то $m$ и $m'$ можно выбрать настолько большими, что равенство будет выполнено с любой точностью, заданной заранее, следовательно физически это \textbf{возможно всегда}.\\
Тогда в результате кругового процесса:
\begin{enumerate}
	\item Состояние нагревателя не изменилось;
	\item $Q_2'-Q_2=(\eta - \eta')Q_1 > 0$ --- отданное холодильником тепло;
	\item $\eta Q_1 -\eta'Q_1'=(\eta-\eta')Q_1>0$ --- совершенная машиной работа. 
\end{enumerate}
Таким образом единственным результатом кругового процесса будет произведение работы $(\eta-\eta')Q_1>0$ за счет эквивалентного количества тепла, заимствованного у холодильника.
Получаем противоречие с постулатом Томпсона--Планка, значит предположение $\eta>\eta'$ --- неверно, аналогично $\eta<\eta$ --- неверно, следовательно $\eta = \eta'$.\\

\textbf{Вторая теорема Карно:} КПД любой тепловой машины, работающей между двумя резервуарами, не может превышать КПД машины Карно, работающей между теми же резервуарами.\\
\textbf{Доказательство.} Пусть машина получает элементарное количество теплоты $\delta Q_1$ и отдает $\delta Q_2$, $T_1$ и $T_2$ --- температуры нагревателя и холодильника остаются постоянными \\
$\int \dfrac{\delta Q_1}{T_1}-\int \dfrac{\delta Q_2}{T_2}\leqslant0 \then \dfrac{Q_1}{T_1}-\dfrac{Q_2}{T_2}\leqslant0,$ где $Q_1$ --- полное количество тепла, полученное от нагревателя, а $Q_2$ --- полное количество тепла, отданное холодильнику. Тогда\\
$\dfrac{Q_2}{Q_1}\geqslant\dfrac{T_2}{T_1}\Leftrightarrow\eta \equiv1-\dfrac{Q_2}{Q_1}\leqslant1-\dfrac{T_2}{T_1}\equiv\eta_\text{Карно}.$
